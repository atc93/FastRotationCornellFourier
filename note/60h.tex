\documentclass{./src/gm2}

% \setlength{\titleblockheight}{10cm}%adjust this lenght if it is needed

\usepackage[english]{babel}
\usepackage[utf8]{inputenc}
\usepackage{amsmath,amssymb,amsfonts,bm}
\usepackage{mathrsfs,overpic}
\usepackage[href]{./src/journals}
\usepackage{lscape,xspace}
\usepackage{subfigure}
\usepackage{morefloats}
\usepackage{booktabs}
\usepackage{adjustbox}
\usepackage{graphicx}

%\usepackage{showframe}   

\newcommand{\overbar}[1]{\mkern+2mu\overline{\mkern-2mu#1\mkern-1mu} \mkern+1mu}
\renewcommand{\epsilon}{\varepsilon}
\renewcommand{\rho}{\varrho}
\newcommand{\p}{\ensuremath{\bm{p}}}
\newcommand{\re}{\ensuremath{\Re {\rm e}}}
\newcommand{\im}{\ensuremath{\Im {\rm m}}}
\newcommand{\mean}[1]{\ensuremath{\langle #1 \rangle}}
\newcommand{\bra}[1]{\ensuremath{\langle#1|}}
\newcommand{\ket}[1]{\ensuremath{|#1\rangle}}
\newcommand{\mup}{\mbox{\ensuremath{\mu^+}}}
\newcommand{\mum}{\mbox{\ensuremath{\mu^-}}}
%\newcommand{\MeVc}{\mbox{\ensuremath{\text{MeV}/c}}}
%\newcommand{\GeVc}{\mbox{\ensuremath{\text{GeV}/c}}}
\newcommand{\geant}{{\tt GEANT}}
\newcommand{\wfd}{{\tt WFD}}
\newcommand{\orcad}{{\tt OrCAD}}
\newcommand{\opera}{{\tt OPERA}}
\newcommand{\mars}{{\tt MARS}}
\newcommand{\gbeamline}{{\tt G4Beamline}}
\newcommand{\ea}{{\em et al.}}
\newcommand{\amu}[1][]{\ensuremath{a_{\mu^{#1}}}}
\newcommand{\gm}{\ensuremath{(g-2)}}
\newcommand{\wa}{\mbox{\ensuremath{\omega_a}}}
\renewcommand{\wp}{\mbox{\ensuremath{\omega_p}}}
\newcommand{\wpt}{\mbox{\ensuremath{\widetilde{\omega}_p}}}
\newcommand{\mus}{\mbox{\ensuremath{\mu\text{s}}}}
\newcommand{\Exp}{\mbox{New \g2\ Experiment}}
\newcommand{\runo}{{\tt RUN~1}}
\newcommand{\sixty}{{\tt 60h}}
\newcommand{\fr}{{Fast~Rotation}}
\newcommand{\esq}{{\tt ESQs}}
\newcommand{\tz}{\ensuremath{t_{0}}}
\newcommand{\tm}{\ensuremath{t_{M}}}
\newcommand{\ts}{\ensuremath{t_{S}}}

%% Physics
\newcommand{\xxx}{\textcolor{green}{XXX}\xspace}

\newtheorem{definition}{Definition}
\newtheorem{theorem}{Theorem}

\begin{document}

%opening
\title{Fast Rotation Fourier analysis of the \sixty\ data-set}
\noteid{G-2 Internal}
\version{1.0}
\author[1]{\mail{antoine.chapelain@cornell.edu}{A.~Chapelain}}
\author[1]{J.~Fagin}
\author[1]{D.~Rubin}
\author[1]{D.~Seleznev}

\affil[1]{Cornell University}

\maketitle

\tableofcontents

\section{Introduction}

This note presents the Cornell \fr\ analysis via the Fourier approach of the so-called $\sixty$ data set collected during $\runo$.
The \fr\ analysis allows to extract the frequency distribution of the stored anti-muon beam or equivalently the momentum or radial distribution.
The knowledge of the radial distribution of the beam is essential in order to estimate the correction to \wa\ from the electro-static quadrupoles (\esq),
the so-called electric field correction.

\subsection{Fast Rotation signal}

The so-called \fr\ signal corresponds to the number of positron counts as a function of time $N(t)$ corrected by the major signal
features so as to leave only the bunching feature of the beam. The features corrected for are: the boosted anti-muon life-time $\tau$,
the spin precession frequency of the anti-muon \wa\; so as to obtain a normalized intensity spectrum whose feature is to a
really good first approximation due to the bunching characteristics of the beam. The correction is performed using the standard
5-parameter fit:

\begin{equation}
        N(t) = N_{0}e^{-t/\tau}[1+Acos(\wa t + \phi)],
\end{equation}

where $N_{0}$ is the number of detected positron at the start of the fit, $A$ called the asymmetry is the amplitude of the
modulation due to the spin precession and $\phi$ the phase of the modulation.

\subsection{Fourier analysis}

The Fourier analysis relies on calculating the cosine (or real part) Fourier transform of the \fr\ signal:

\begin{equation}
    S(\omega,t_{s},t_{m}) = \int^{t_{m}}_{t_{s}} S(t) cos\omega(t-t_{0})dt,
\label{eq:cosine}
\end{equation}

where $S(t)$ is the \fr\ signal, \tz\ the time corresponding to the center of mass of the beam passing the detector for the first time,
and \ts\ and \tm\ are respectively the start and end time of the \fr\ signal. The parameter \tz, \ts\ and \tm\ are at the core of the
Fourier analysis. In the ideal case of having access to the data from the first turn ($t_{s}=t_{0}$) when the anti-muon beam enters the 
ring, the equation above is exact and yield to the correct frequency distribution $\Phi(\omega)$: 

\begin{equation}
    \Phi(\omega) = S(\omega,t_{s}=t_{0},t_{m}) = \int^{t_{m}}_{t_{0}} S(t) cos\omega(t-t_{0})dt
\end{equation}

Unfortunately the calorimeter data for \runo\ are only
available after couple micro-seconds. In this case eq. (\ref{eq:cosine}) is only a good approximation of the frequency distribution and it needs an additional
correction term, which is:

\begin{equation}
    \Delta(\omega)=\int^{\omega^+}_{\omega^-}{S}(\omega')\frac{\sin(\omega-\omega')(t_s-t_0)}{\omega-\omega'}d\omega'
    \label{eq:OrlovDelta}
\end{equation}

The correct frequency distribution can be retrived via:

\begin{equation}
    \Phi(\omega) = \int^{t_{m}}_{t_{s}} S(t) cos\omega(t-t_{0})dt + A \cdot \int^{\omega^+}_{\omega^-}{S}(\omega')\frac{\sin(\omega-\omega')(t_s-t_0)}{\omega-\omega'}d\omega' + B,
\end{equation}

where $A$ and $B$ are optimized so that the frequency distribution vanishes in the regions outside the collimators aperture where muons cannot be stored.
For more details regarding the Fourier approach, please refer to~\cite{orlov, daniel}.


\section{Producing the Fast Rotation signal}


\section{\tz\ optimization}

fds


\section{Frequency and radial distribution}


\section{E-field correction}



\section{Statistical uncertainty}


\section{Systematic uncertainty}

\subsection{\tz}

scan around minimum
jump by couple FR period

\subsection{\ts}

fr period effect
use a maximum or minimum of fr?

\subsection{\tm}

results pretty steady after 100 micro-secs

\subsection{Energy threshold}
\subsection{Calorimeters timing alignement}
\subsection{Field index}
\subsection{Frequency to radius conversion}

need betatron motion
assume non-uniform velocity

\subsection{Binning effect}

frs
fourier transform
wiggle fit

\subsection{E-field correction approximations}


\begin{thebibliography}{}

\bibitem{orlov}
Y. Orlov et al., NIM A 482 (2002) 767-755.

\bibitem{daniel}
A. Chapelain, D. Rubin, D. Seleznev, \textit{Extraction of the Muon Beam Frequency Distribution via the Fourier Analysis of the Fast Rotation Signal}, E989 note 130, 
GM2-doc-9701

\end{thebibliography}


\end{document}
